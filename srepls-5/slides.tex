\documentclass{beamer}
%% apparently this magic helps avoid the dreaded
%% ``Too many math alphabets used in version normal''
%% error. Yuk.
\newcommand\hmmax{0}
\newcommand\bmmax{0}
%% end magic

\usepackage{amssymb,amsmath,amsthm}
\usepackage{latexsym}

\usepackage{array}

\usepackage{url}
\usepackage{hyperref}

%% uses the AMS Euler math font.
\usepackage{euler}

%% for coloneqq
%% \usepackage{mathtools}

%% %% for \underaccent
%% \usepackage{accents}

% for semantic brackets
%% \usepackage{stmaryrd}

% for inference rules
\usepackage{proof}

%% for mathpar environment
\usepackage{mathpartir}

% for \scalebox, \rotatebox
%% \usepackage{graphicx}

%% for colors
\usepackage{color}

%% for drawing Hasse diagram
% \usepackage{tikz}


\newcommand{\ms}{\mathsf}
\newcommand{\mb}{\mathbf}
\newcommand{\x}{\times}
\newcommand{\setfor}[2]{\{#1 ~|~ #2\}}


\title{Finding fixed points faster}
\author{Michael Arntzenius}
\institute{University of Birmingham}
\date{HOPE @ ICFP 2018}

\begin{document}
\maketitle

\begin{frame}
  \huge
  \begin{center}
    \begin{tabular}{ccc}
      %% Datalog &$\subseteq$& Datafun\vspace{0.5em}\\
      \parbox{4cm}{
        \begin{center}
          \uncover<2>{\textsuperscript{1}} Datalog\\ +\\
          \uncover<2>{\textsuperscript{2}} semi-na\"ive\\\ \ evaluation
        \end{center}}
      %% semi-na\"ive evaluation
      &{\Huge $\subseteq$}&
      \parbox{4cm}{
        \begin{center}
          \uncover<2>{\textsuperscript{3}} Datafun\\ +\\
          \uncover<2>{\textsuperscript{4}} incremental $\lambda$-calculus
        \end{center}
      }
    \end{tabular}
  \end{center}
\end{frame}


%% \begin{frame}
%%   \huge\centering\bf Finding fixed points faster by static incrementalization
%% \end{frame}



\begin{frame}
  \Huge \centering \textbf{Datalog}\vspace{0.66cm}

  \huge decidable logic programming\vspace{0.66cm}

  predicates = finite sets
\end{frame}

\begin{frame}[fragile]
  \Large
  \begin{center}
\begin{verbatim}
% Transitive closure of `edge'.
path(X,Y) :- edge(X,Y).
path(X,Z) :- edge(X,Y), path(Y,Z).
\end{verbatim}
  \end{center}
\end{frame}


%% The naive implementation
\begin{frame}{\Large Na\"ive implementation}\large
  \[
  \begin{array}{l}
    \text{step} ~:~ \ms{Set}\; (\ms{Node} \x \ms{Node}) \to
    \ms{Set}\; (\ms{Node} \x \ms{Node})\\
    \text{step}\; path =
    {\color<2->{red} \setfor{(x,y)}{(x,y) \in edge}}\\
    \phantom{\text{step}\;path} \hspace{1.15pt}\cup\hspace{1.15pt}
    {\color<2->{orange} \setfor{(x,z)}{(x,y) \in edge, (y,z) \in path}}\\
    \\
    \text{fix} ~:~ (\alpha \to \alpha) \to \alpha \to \alpha\\
    \text{fix} \; f \; current =
    \mathbf{let}~ next = {\color<2->{red}f \; current} ~\mathbf{in}\\
    \phantom{\text{fix} \; f \; current ={ }}
    \mathbf{if}~next = current ~\mathbf{then}~ current\\
    \phantom{\text{fix} \; f \; current ={ }}
    \mathbf{else}~ \text{fix}\; f\; next\\
    \\
    \text{path} ~:~ \ms{Set} \; (\ms{Node} \x \ms{Node})\\
    \text{path} = \text{fix} \; \text{step} \; \emptyset
  \end{array}
  \]\vspace{0em}

  \uncover<2->{\Huge \centering \color{red}  Unnecessary recomputation!}
\end{frame}


\begin{frame}{\Large Semi-na\"ive implementation}\large
  \[
  \begin{array}{l}
    \textsf{small-step} ~:~ \ms{Set}\;(\ms{Node} \x \ms{Node})
    \to \ms{Set}\;(\ms{Node} \x \ms{Node})\\
    \textsf{small-step} \; new =
    \setfor{(x,z)}{(x,y) \in edges, (y,z) \in new}\\
    \\
    \pause
    \textsf{fix-faster} : (\ms{Set}\,\alpha \to \ms{Set}\,\alpha \to \ms{Set} \,\alpha)
    \to \ms{Set}\,\alpha \to \ms{Set}\,\alpha \to \ms{Set}\,\alpha\\
    \textsf{fix-faster} \; f \; current \; new =\\
    \quad \mathbf{let}~\textit{to-add} = f \; current \; new ~\mathbf{in}\\
    \quad \mathbf{if}~ \textit{to-add} \subseteq current ~\mathbf{then}~ current\\
    \quad \mathbf{else}~ \textsf{fix-faster}\; f \; (current \cup \textit{to-add}) \; \textit{to-add}\\
    \\
    \pause
    \textsf{path} ~:~ \ms{Set}\; (\ms{Node} \x \ms{Node})\\
    \textsf{path} = \textsf{fix-faster}\; (\lambda x\, dx.\; \textsf{small-step}\; dx)
    \; edge \; edge
  \end{array}
  \]
\end{frame}


\begin{frame}
  \Huge\centering
  $\ms{fix} : (\alpha \to \alpha) \to \alpha \to \alpha$

  \vspace{0.66em}
  \huge
  as iteration, not laziness
\end{frame}

\begin{frame}
  \Huge 
  \begin{center} Datafun \end{center}

  \huge
  \begin{itemize}
  \item Simply-typed $\lambda$-calculus
  \item with iterative fixed points
  \item and monotonicity typing
  \end{itemize}
  \vspace{0.5cm}

  \large (and other stuff, see ICFP'16 paper)
\end{frame}

\begin{frame}[fragile]
  \Large\textbf{Datalog:}
\begin{verbatim}
path(X,Y) :- edge(X,Y).
path(X,Z) :- edge(X,Y), path(Y,Z).
\end{verbatim}
\vspace{0.4cm}

\Large\textbf{Datafun:}
  \[
  \begin{array}{l}
    \mb{fix}~path~\mb{is}~edge\\
    \phantom{\mb{fix}~path}\,\cup~\setfor{(x,z)}{(x,y) \in edge, (y,z) \in path}
  \end{array}
  \]
\end{frame}


\begin{frame}
  \Large
  Na\"ive implementation strategy for
  \[
  \begin{array}{l}
    \mb{fix}~path~\mb{is}~edge\\
    \phantom{\mb{fix}~path}\,\cup~\setfor{(x,z)}{(x,y) \in edge, (y,z) \in path}
  \end{array}
  \]

  is
  \[
  \begin{array}{l}
    \textsf{fix}\; (\lambda\, path.\\
    \phantom{\textsf{fix}\; (\lambda\, } edge \cup \setfor{(x,z)}{(x,y) \in edge, (y,z) \in path})
    \\ \phantom{\textsf{fix}\;}\emptyset
  \end{array}
  \]

  using our iterative `$\mathsf{fix}$' function from earlier.
  \vspace{1em}

  \textbf{Is there an analogue of {\normalfont faster-fix}?}
\end{frame}


\begin{frame}
  \Huge \centering
  Incremental $\lambda$-Calculus\vspace{0.5cm}

  \large ``A Theory of Changes for Higher-Order Languages'', PLDI'14\\Yufei
  Cai, Paulo Giarrusso, Tillman Rendel, Klaus Ostermann
\end{frame}

\begin{frame}
  \Large For every type $A$
  \begin{itemize}
  \item a \emph{change type} $\Delta A$ 
  \item and operator $\oplus : A \to \Delta A \to A$.
  \end{itemize}
  \vspace{0.5cm}

  Given term $f : A \to B$\\
  \begin{itemize}
  \item $\delta f : A \to \Delta A \to \Delta B$
  \item such that $f(x \oplus dx) = f\; x \oplus \delta f\; x\; dx$
  \end{itemize}
  if one knows $v = f\; x$, often cheaper to compute RHS!
\end{frame}

\begin{frame}
  \centering {\huge
  $(\lambda x\, dx.\; \textsf{small-step}\;dx) \approx \delta(\textsf{step})$~{\bf !}}

  \pause
  \large
  \[\begin{array}{l}
    \text{step}\; path =
    \setfor{(x,y)}{(x,y) \in edge}\\
    \phantom{\text{step}\;path} \hspace{1.15pt}\cup\hspace{1.15pt}
    \setfor{(x,z)}{(x,y) \in edge, (y,z) \in path}\\
    \\
    \textsf{small-step} ~:~ \ms{Set}\;(\ms{Node} \x \ms{Node})
    \to \ms{Set}\;(\ms{Node} \x \ms{Node})\\
    \textsf{small-step} \; new =
    \setfor{(x,z)}{(x,y) \in edges, (y,z) \in new}\\
  \end{array}\]
  \vspace{0cm}

  \pause
  \huge\bf Find fixed points faster by static incrementalization!
\end{frame}


\begin{frame}
  \large
  \[\begin{array}{l}
  \textsf{faster-fix} ~:~ (\alpha \to \Delta \alpha \to \Delta \alpha)
  \to \alpha \to \Delta \alpha \to \alpha\\
  \textsf{faster-fix} \; df \; current \; change =\\
  \quad \mb{let}~ next = current \oplus change ~\mb{in}\\
  \quad \mb{if}~ next \le current ~\mb{then}~ current\\
  \quad \mb{else}~ \textsf{faster-fix} \; df \; next \; (df \; current \; change)
  \end{array}\]

  %% \vspace{0.5cm}

  %% Then $(\mb{fix}~x~\mb{is}~f\; x)$ becomes
  %% $(\textsf{faster-fix} \; \delta f \; \emptyset \; (f\; \emptyset))$

  \vspace{0.5cm}

  \small (I have a proof in my notes that
  this slide is too small to contain.)
\end{frame}

\begin{frame}{Applying this to Datafun}
  \large
  \begin{itemize}
  \item Monotonicity $\to$ increasing changes only!\\
    $\Delta(\ms{Set}\; \alpha) = \ms{Set}\;\alpha$

    \vspace{0.2cm}

  \item $\Delta(\square A) = 1$? No!

    \textbf{Zero-changes are not trivial!}

    \vspace{0.2cm}

  \item $\delta (\bigvee(x \in e_1)\; e_2)$?

    In particular, if $e_1 : \ms{Set}\; (A \to B)$.

    \vspace{0.2cm}

  \end{itemize}
\end{frame}

\end{document}
