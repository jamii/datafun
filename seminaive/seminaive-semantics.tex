\documentclass[nomarginums]{rntz}
\usepackage[b5]{rntzgeometry}
%\usepackage[phone]{fantasy}
\usepackage[baskerville,newmath]{rntzfont}
\usepackage{anyfontsize}
\usepackage[spacing=true,stretch=10]{microtype}
\frenchspacing

\usepackage{amssymb,amsmath,amsthm} % \square etc.
%\usepackage{array}                  % >{blah}, <{blah} in array formats.
%\usepackage{booktabs}               % \midrule
%\usepackage{mathpartir}             % \begin{mathpar}, \infer, etc.
\usepackage{stmaryrd}               % \shortrightarrow, \llbracket, etc.
\usepackage{nccmath}                % fix align* (etc.) spacing

%% Commands
\newcommand\todo[1]{{\color{Rhodamine}#1}}

\newcommand\naive{na\"ive}
\newcommand\Naive{Na\"ive}
\newcommand\cat\textbf
\newcommand\strong\textbf
\newcommand\CP{\cat{ChangePoset}}
\newcommand\Poset{\cat{Poset}}
\newcommand\initObj{\ensuremath{\mathbold{0}}}
\newcommand\termObj{\ensuremath{\mathbold{1}}}

\newcommand\G\Gamma
\newcommand\D\Delta
\newcommand\x\times
\newcommand\dee\delta
\newcommand\tuple[1]{\left({#1}\right)}
\newcommand\triv{\star}
\newcommand\injc{\mathrm{in}}
\newcommand\inj[1]{\injc_{#1}\,}

\newcommand\iso{\texorpdfstring{\ensuremath{\square}}{iso}}
\newcommand\isof[1]{\iso {#1}}
\newcommand\fname[1]{\textit{#1}}
\newcommand\id{\fname{id}}
\newcommand\dummy{\fname{dummy}}

%% TODO: look at other latex arrows
\newcommand\validarrow{{\to}}
\newcommand\longvalidarrow{{\longrightarrow}}
\newcommand\valid[1]{\mathrel{\overset{#1}{\validarrow}}}
\newcommand\longvalid[1]{\mathrel{\overset{#1}{\longvalidarrow}}}
\newcommand\vals[1]{#1^v} % other options: _v, _o, _\iota
\newcommand\chgs[1]{\D{#1}}

\newcommand\fork[1]{\langle{#1}\rangle}
\newcommand\funct[1]{\vals{#1}}
\newcommand\deriv[1]{#1'}

\newcommand\bindsp{~\,}
\newcommand\fa[1]{\forall #1.\bindsp}
\newcommand\ex[1]{\exists #1.\bindsp}
%% \renewcommand\ex[1]{(\exists #1)\bindsp}
%% \renewcommand\fa[1]{(\forall #1)\bindsp}


\title{Change Semantics for Semi\naive{} Datafun}
\author{Michael Arntzenius}
\date{5 November 2018}

\begin{document}

\maketitle

\begin{abstract}
  We define the category \CP{} of posets equipped with \emph{increasing
    changes}, whose morphisms are monotone maps equipped with \emph{derivatives}
  taking input changes to output changes. We give a semantics for Datafun in
  \CP.
\end{abstract}

\todo{TODO: should I be using posets or preorders?}


\section{The discreteness comonad \iso{} on \Poset{}}
\todo{TODO: Explain $\iso$.}


\section{The category \CP}

Objects $A$ of \CP{} are tuples $(\vals A, \chgs A, \validarrow_A)$, where
\begin{enumerate}
\item $\vals A \in \Poset$ is the poset of values.
\item $\chgs A \in \Poset$ is the poset of changes.
\item $x \valid{dx} y : A$ is a relation over $x,y : \vals A$ and $dx : \chgs
  A$, glossed as ``$dx$ changes $x$ into $y$''. We will also call this
  ``validity'', saying $dx$ is a \emph{valid} change to $x$ iff $\ex{y} x
  \valid{dx} y$.
\end{enumerate}

\noindent They must satisfy the following conditions:\footnote{There are other
  conditions one might sensibly impose here. For example, a composition operator
  such that if $x \valid{dx} y$ and $y \valid{dy} z$ then $x \longvalid{dx
    \cdot dy} z$; or a relation between the ordering on values and the ordering
  on changes, \(x \valid{dx_1} y_1 \wedge x \valid{dx_2} y_2 \implies (y_1 \le
  y_2 \iff dx_1 \le dx_2)\). I have omitted these because I have not needed
  them, but the lack of an obvious ``correct'' set of conditions bothers me.}
\begin{align*}
  x \le y : \vals A &~\implies \ex{dx} x \valid{dx} y
  & \text{complete for increases}
  \\
  x \le y : \vals A &\impliedby~ \ex{dx} x \valid{dx} y
  & \text{sound for increases}
\end{align*}

\noindent
A morphism $f \in \CP(A, B)$ is a value map $\funct f \in \Poset(\vals A,
\vals B)$ equipped with a derivative $\deriv f \in \Poset(\isof{\vals A} \x
\chgs A, \chgs B)$ such that:

\[x \valid{dx} y \implies \funct f\,x \longvalid{\deriv f(x,dx)} \funct f\,y\]

\noindent
Two morphisms are equal iff they have equal value maps and their derivatives'
behavior coincides on \emph{valid} changes. In other words, we \strong{quotient}
equality of derivatives as follows:

\[ (\fa{x \valid{dx} y} f'(x,dx) = g'(x,dx)) \implies f' = g' \]

\noindent
Composition follows the chain rule:

\begin{align*}
  \funct\id\, x &= x &
  \funct{(f \circ g)}\, x &= \funct f (\funct g\,x)\\
  \deriv\id(x,dx) &= dx
  & \deriv{(f \circ g)} (x,dx) &= \deriv f (g \;x,\, \deriv g(x,dx))
\end{align*}

\noindent The only interesting case of associativity is the one involving the
chain rule:
\begin{align*}
  \deriv{(f \circ (g \circ h))} (x,dx)
  &= \deriv f((g \circ h) \;x,\ \deriv{(g \circ h)}(x,dx))\\
  &= \deriv f (g(h \;x),\ \deriv g(h \;x,\ \deriv h(x,dx)))\\
  &= \deriv{(f \circ g)} (h \;x,\ \deriv h(x,dx))\\
  &= \deriv{((f \circ g) \circ h)} (x,dx)
\end{align*}

\todo{TODO: Find a good place to note that $\validarrow$ is usually (partially)
  functional over first-order data, but fails to be over functions.}


\subsection{\CP{} refines \Poset{}}
\label{sec:refines}

There is a straightforward forgetful functor $\vals{-} : \CP \to \Poset$ taking
objects $A$ to $\vals{A}$ and morphisms $f$ to $\funct f$. The structures we
will build in \CP{} all \emph{refine} the corresponding structures in \Poset{},
meaning they commute with $\vals{-}$. For example, for cartesian products,
$\vals{(A \x B)} = \vals{A} \x \vals{B}$ and $\funct{\fork{f,g}} = \fork{\funct
  f, \funct g}$. For brevity's sake, we omit the ``value components'' of our
constructions when they can be derived from this fact.


\subsection{Cartesian structure}

\begin{theorem}
  \CP{} has all finite products and sums.
\end{theorem}

\begin{proof}
  We show this by constructing initial and terminal objects and binary sums and
  products. In each case, both the underlying and the change posets come from
  the corresponding structure in \Poset{}:

  \begin{align*}
    \vals \initObj = \chgs \initObj &= \initObj \in \Poset &
    \vals \termObj = \chgs \termObj &= \termObj \in \Poset \\
    \vals{(A \x B)} &= \vals A \x \vals B &
    \vals{(A + B)} &= \vals A + \vals B \\
    \chgs{(A \x B)} &= \chgs A \x \chgs B &
    \chgs{(A + B)} &= \chgs A + \chgs B
  \end{align*}

  \noindent The validity relationships are given inductively by:

  \[\setlength\arraycolsep{.25em}\def\arraystretch{1.45}
  \begin{array}{rcl}
    \multicolumn{3}{c}{\triv \valid{\triv} \triv : \termObj} \\
    a \valid{da} a' \wedge b \valid{db} b'
    &\implies& \tuple{a,b} \longvalid{\tuple{da,db}} \tuple{a',b'} \\
    x \valid{dx} y &\implies&
    \inj{i} x \longvalid{\inj{i} dx} \inj{i} y
  \end{array}\]

  \noindent Here, $\triv$ is the single inhabitant of $\termObj \in \Poset$. We
  give no cases for $\initObj$ because it is empty.

  The ``value components'' of the morphisms $!$, \textexclamdown, $\pi_i$,
  $\injc_i$, $\fork{f, g}$, and $[f,g]$ are inherited from \Poset{} along
  $(\vals{-})$ as mentioned in \cref{sec:refines}. The derivatives of $!$,
  \textexclamdown, $\pi_i$, and $\injc_i$ simply operate pointwise on their
  second argument:

  \begin{align*}
    {\deriv{!}}(x,dx) &= {\funct{!}}\,dx &
    \deriv{\text{\textexclamdown}}(x,dx) &= \funct{\text{\textexclamdown}}\,dx\\
    \deriv\pi_i(x,dx) &= \funct{\pi_i}\,dx &
    \deriv\injc_i(x,dx) &= \funct{\injc_i}\,dx
  \end{align*}

  \noindent
  Note that $!$ and {\textexclamdown} inherit their universal properties from
  \Poset{}.

  We can calculate $\deriv{\fork{f_1,f_2}}$ from its universal property.
  Consider some $g$ such that $\deriv f_i = \deriv{(\pi_i \circ g)}$ for $i \in
  \{1,2\}$. Then:

  \begin{align*}
    \deriv f_i(x,dx)
    &= \deriv{(\pi_i \circ g)}(x,dx)\\
    &= \deriv\pi_i (\funct g\,x,\, \deriv g(x,dx))\\
    &= \funct\pi_i(\deriv g(x,dx))
  \end{align*}

  \noindent
  So by the universal property of products in \Poset{}, we have $g' =
  \fork{\deriv f_1, \deriv f_2}$. Thus $\deriv{\fork{f_1,f_2}} = \fork{\deriv
    f_1, \deriv f_2}$.

  Finally, let's attempt to calculate $\deriv{[f_1,f_2]}$ from its universal
  property. Consider some $g$ such that $\deriv f_i = \deriv{(g \circ
    \injc_i)}$. Then:

  \begin{align*}
    \deriv f_i(x,dx)
    &= \deriv{(g \circ \injc_i)}(x,dx)\\
    &= \deriv g (\funct\injc_i \,x,\, \deriv\injc_i \,(x,dx))\\
    &= \deriv g (\inj i x,\, \inj i dx)
  \end{align*}

  \noindent
  This defines $g'(\inj i x,\, \inj j dx)$ when $i = j$. What about $i \ne j$?
  Here we appeal to our quotient: derivatives are equal iff they agree on
  \emph{valid} changes. By construction of the validity relation for sums, all
  valid changes to $\inj i x$ are of the form $\inj i dx$. So it \strong{does
    not matter} what $\deriv{[f_1,f_2]}(\inj i x, \inj j dx)$ does for $i \ne
  j$, so long as it is defined. One satisfactory definition is:

  \begin{equation*}
    \deriv{\left[f_1,f_2\right]} (\inj{i} x,\, \inj{j} dx)
    = \begin{cases}
      \deriv f_i(x,dx) & \text{if}~ i=j\\
      \fname{magic}(x) & \text{otherwise}
    \end{cases}
  \end{equation*}

  \noindent
  Here $\fname{magic} : \vals A \to \chgs A$ is given by the axiom of choice as
  follows. Fix $x : \vals A$. By reflexivity $x \le x$, so by soundness,
  $\exists dx : \chgs A$ (indeed, one such that $x \valid{dx} x$, but this is
  irrelevant here). Let \fname{magic} be a choice function that picks such a
  $dx$.

  I omit the proofs that the derivatives are monotone in their second argument.
\end{proof}

\noindent
NB. \fname{magic} is effectively dead code: necessary to define $\deriv{[f,g]}$,
but irrelevant to its behavior. We will exploit this later to avoid needing the
axiom of choice in the context of Datafun.


\subsection{Exponential objects}

\newcommand\expObj[2]{{#1} \shortrightarrow {#2}}
%% fucking kerning.
\newcommand\df{d\kern-0.1em f\kern-.1em}

The change poset of the exponential object $\expObj{A}{B} \in \CP$ consists of
monotone maps $\df, dg \in \expObj{\isof{\vals A}}{\expObj{\chgs A}{\chgs B}}$,
partially ordered pointwise by their effect on valid changes and quotiented to
satisfy antisymmetry:

\[ \df \le dg \iff \fa{x \valid{dx} y} \df \;x \;dx \le dg \;x \;dx \]

\noindent
The validity relation is:

\[ f \valid{\df} g \iff \fa{x \valid{dx} y} f\;x \longvalid{df \;x \;dx} g\;y
\]


\subsection{The lifted \iso{} comonad}

\subsection{Semilattice objects}

\subsection{Fixed points}

\end{document}
