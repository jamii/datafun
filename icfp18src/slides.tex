\documentclass[xcolor={dvipsnames}]{beamer}

\usefonttheme{serif}
\usepackage[pteuler]{rntzfont} % pteuler, charter work well
%\usepackage{PTSans}
\usepackage[semibold]{sourcesanspro}

%\usepackage{booktabs}
\usepackage{anyfontsize} % fix font warnings from stmaryrd
\usepackage{latexsym,amssymb,stmaryrd,mathtools}
\usepackage{mathpartir}
\usepackage{multirow}
\usepackage{tikz,tikz-cd}       % Hasse & commutative diagrams.
\usepackage{url,hyperref}

\usepackage[normalem]{ulem}               % for \sout
\newcommand{\msout}[1]{\text{\sout{\ensuremath{#1}}}}

\definecolor[named]{ACMBlue}{cmyk}{1,0.1,0,0.1}
\definecolor[named]{ACMYellow}{cmyk}{0,0.16,1,0}
\definecolor[named]{ACMOrange}{cmyk}{0,0.42,1,0.01}
\definecolor[named]{ACMRed}{cmyk}{0,0.90,0.86,0}
\definecolor[named]{ACMLightBlue}{cmyk}{0.49,0.01,0,0}
\definecolor[named]{ACMGreen}{cmyk}{0.20,0,1,0.19}
\definecolor[named]{ACMPurple}{cmyk}{0.55,1,0,0.15}
\definecolor[named]{ACMDarkBlue}{cmyk}{1,0.58,0,0.21}

\title{Type inference for monotonicity}
\author{Michael Arntzenius}
\institute{University of Birmingham}
\date{S-REPLS 9, 2018}

\newcommand{\R}{\mathbb{R}}
\newcommand{\N}{\mathbb{N}}
\newcommand{\x}{\times}
\newcommand{\G}{\Gamma}
\newcommand{\fn}{\lambda}
\newcommand{\binder}{.~}
\newcommand{\bind}[1]{{#1}\binder}
\newcommand{\fnof}[1]{\fn\bind{#1}}
\newcommand{\den}[1]{\llbracket{#1}\rrbracket}
\newcommand{\bnfeq}{\dblcolon=}
\newcommand{\hilit}{\color{Rhodamine}}
\newcommand{\hil}[1]{{\hilit#1}}

\newcommand{\rulename}{\sffamily\scshape}

\newcommand{\mb}[1]{\ensuremath{\textbf{#1}}}
\newcommand{\mi}[1]{\ensuremath{\textit{#1}}}
\newcommand{\kw}[1]{\mb{#1}}

\newcommand{\id}{\text{id}}
\newcommand{\op}{\text{op}}
\newcommand{\iso}{\ensuremath{\Box}}
\renewcommand{\path}{\ensuremath{\lozenge}}

\newcommand{\idof}{\mathop{\id}}
\newcommand{\opof}{\mathop{\op}}
\newcommand{\isof}{\iso}
\newcommand{\pathof}{\path}


\begin{document}
\maketitle
\Large

\begin{frame}
  \Huge\centering
  %% Which ones are monotone?
  \[\begin{array}{c}
  x + \log x\vspace{.5em}\\
  -x \vspace{.5em}\\
  x - \log x \vspace{.5em}\\
  4
  \end{array}\]

  %% How did we know that so quickly?
  %% Did we do a proof in our head? No!
  %% We used compositional reasoning!
  %%
  %% I want a type system that tells me the first one is monotone and the second
  %% isn't.
\end{frame}

\begin{frame}%{Types for monotonicity}
  \begin{mathpar}
    \begin{array}{llcll}
      \text{types}& A,B &\bnfeq& \R ~~|~~ A \to B
      \\ \text{terms}& M,N &\bnfeq& x ~~|~~ \fnof{x} M ~~|~~ M\;N
      \vspace{1em}\\
      \multicolumn{2}{r}{\den{A}} &\in& \mathbf{Poset}\vspace{.2em}\\
      \multicolumn{2}{r}{\den{\R}} &=& \R\\
      \multicolumn{2}{r}{\den{A \to B}}
      &=& \text{\emph{monotone} maps}~ \den{A} \to \den{B},\\
      &&& \text{ordered pointwise}
    \end{array}
  \end{mathpar}
\end{frame}

\begin{frame}
  \huge
  %% TODO: color-hilight these things in turn.
  \[\begin{array}{l@{\hskip 1em}l@{\hskip 1em}l}
    \fnof{x} x + \log x &:& \R \to \R\vspace{.5em}\\
    \fnof{x} {-x} &:&
         \alt<2->{\color{Green}\opof\R}{????\hspace{3.5pt}} \to \R \vspace{.5em}\\
    \fnof{x} x - \log x &:&
         \alt<2->{\isof\R}{???\hspace{3pt}} \to \R \vspace{.5em}\\
    \fnof{x} 4 &:& \alt<2->{\pathof\R}{???\hspace{1.33pt}} \to \R
  \end{array}\]
\end{frame}

\begin{frame}
  %% this would work better with a diagram.
  %%
  %% I should explain that \iso is a comonad,
  %% axioms: □A -> A; □A → □□A
  %% but most modal type theories need further annotations
  %%
  %% explain Pfenning-Davies annotations here?

  Let $\isof{A}$ be $A$, ordered \emph{discretely}:
  \[x \le y : {\isof{A}} \iff x = y : A \]
  \vspace{0pt}

  Then \hil{$f : \isof{A} \to B$ is monotone} iff
  \[x = y \implies f(x) \le f(y) \]
  i.e. \textbf{always}!
\end{frame}

\begin{frame}
  \huge
  %% TODO: color-hilight these things in turn.
  \[\begin{array}{l@{\hskip 1em}l@{\hskip 1em}l}
    \fnof{x} x + \log x &:& \R \to \R\vspace{.5em}\\
    \fnof{x} {-x} &:&
         {\opof\R} \to \R \vspace{.5em}\\
    \fnof{x} x - \log x &:&
         \color{Green}{\isof\R \to \R} \vspace{.5em}\\
    \fnof{x} 4 &:& {\pathof\R} \to \R
  \end{array}\]
\end{frame}

%% \begin{frame}
%%   \huge
%%   \[\begin{array}{l}
%%     (\fnof{x} x + \log x) : \R \to \R\vspace{.5em}\\
%%     (\fnof{x} x - \log x) : {\color{Green} \isof{\R}} \to \R
%%   \end{array}\]
%% \end{frame}

\begin{frame}
  $\pathof{A}$ \emph{identifies} weakly connected elements in $A$:
  \[ x \le y \vee y \le x : A \implies x = y : \pathof{A} \]
  \vspace{0pt}

  \pause
  \textbf{Theorem:}
  \[ f : \pathof{A} \to B \iff f : A \to \isof{B} \]
\end{frame}

\begin{frame}
  \huge
  \[\begin{array}{l@{\hskip 1em}l@{\hskip 1em}l}
    \fnof{x} x + \log x &:& \R \to \R\vspace{.5em}\\
    \fnof{x} {-x} &:&
         {\opof\R} \to \R \vspace{.5em}\\
    \fnof{x} x - \log x &:&
         {\isof\R \to \R} \vspace{.5em}\\
    \fnof{x} 4 &:& \color{Green} \pathof\R \to \R
  \end{array}\]
\end{frame}


\begin{frame}
  \iso{} is a \emph{necessity modality / monoidal comonad};\\
  \path{} is a \emph{possibility modality / monoidal monad}.
  \vspace{1em}

  Pfenning \& Davies gave typing rules for these in\\
  \emph{A Judgmental Reconstruction of Modal Logic}!
\end{frame}

\begin{frame}
  \LARGE
  \[
  \begin{array}{l}
    {\color{Red} (\fnof{x} x - \log x) : \isof{\R} \to \R}
    \vspace{.5em}\pause\\
    {(\fnof{x} \kw{let}~\kw{box}~u = x~\kw{in}~
      u - \kw{box}\,(\log u)) : \isof{\R} \to \R}
  \end{array}
  \]
  \vspace{0em}

  \centering\Huge\sffamily :~(
\end{frame}

\begin{frame}
  \textbf{How to infer monotonicity:}

  \begin{enumerate}
  \item Infer variable usage modes bottom-up.
  \item Use modal subtyping for implicit coercion.
  \end{enumerate}
\end{frame}


%% \begin{frame}
%%   \iso{} is a \emph{comonad} or \emph{necessity modality}.

%%   %% \[
%%   %% \begin{array}{rcl}
%%   %%   \isof A &\to& A\\
%%   %%   \isof A &\to& \isof{\isof A}\\
%%   %%   A \to B &\implies& \isof{A} \to \isof{B}
%%   %% \end{array}
%%   %% \]
%%   %% \vspace{0pt}

%%   \vspace{1em}

%%   Pfenning \& Davies gave typing rules for these in\\
%%   \emph{A Judgmental Reconstruction of Modal Logic}!
%% \end{frame}

%% \begin{frame}
%%   \LARGE
%%   \[
%%   \begin{array}{l}
%%     {\color{Red} (\fnof{x} x - \log x) : \isof{\R} \to \R}
%%     \vspace{.5em}\pause\\
%%     {(\fnof{x} \kw{let}~\kw{box}~u = x~\kw{in}~
%%       u - \kw{box}\,(\log u)) : \isof{\R} \to \R}
%%   \end{array}
%%   \]
%%   \vspace{0em}

%%   \centering\Huge\sffamily :~(
%% \end{frame}

%% \begin{frame}{Get rid of pesky annotations in three easy steps!}
%%   \begin{enumerate}
%%   \item Exploit features of \mb{Poset}.
%%   \item Infer variable usage modes bottom-up.
%%   \item Use moded subtyping for implicit coercion.
%%   \end{enumerate}
%% \end{frame}


%% ---- Exploiting Poset ---
%% Include this once we need to explain \path.
%% \begin{frame}{1. Exploiting \textbf{Poset}}
%%   \LARGE%\vspace{-1em}
%%   %% Which ones are monotone?
%%   \[\begin{array}{lcl}
%%   \fnof{x} x + \log x &:& \R \to \R\vspace{.5em}\\
%%   \fnof{x} x - \log x &:& \isof{\R} \to \R \vspace{.5em}\\
%%   \fnof{x} {-x}
%%   &:& \alt<2->{\opof{\R}}{\hil{????\hspace{2.9pt}}} \to \R
%%   \vspace{.5em}\\ \fnof{x} 4 &:&
%%   {\alt<2->{\hilit}{}????\hspace{2.9pt}} \to \R
%%   \end{array}\]
%% \end{frame}


%% \begin{frame}{1. Exploiting \textbf{Poset}}
%%   \LARGE%\vspace{-1em}
%%   %% Which ones are monotone?
%%   \[\begin{array}{lcl}
%%   \fnof{x} x + \log x &:& \R \to \R\vspace{.5em}\\
%%   \fnof{x} x - \log x &:& \isof{\R} \to \R \vspace{.5em}\\
%%   \fnof{x} {-x}
%%   &:& \opof{\R} \to \R \vspace{.5em}\\
%%   \fnof{x} 4
%%   &:& {\color{Green}\pathof{\R}} \to \R
%%   \end{array}\]
%% \end{frame}

%% Maybe I need another frame here?


%% ---- 2. PROPAGATING VARIABLE MODES ----
%% #1: Explain how I'm ferrying tone annotations around on variables.
%%
%% A typical typing judgment Γ ⊢ M : A records
%% - the types of variables used
%% - the type of the term
%%
%% I also want to know: how were those variables *used*?
%% Were they used monotonically, antitonically, discretely, path-ly?
%%
%% #2: Explain what "bottom-up" means.
%%
%% maybe I could explain what I mean using a counterexample?
%% and an example?
\newcommand{\m}[1]{\left[#1\right]}
\newcommand{\disc}[1]{\m{\iso}{#1}}

\begin{frame}{1. Bottom-up mode inference}
  \LARGE
  \begin{mathpar}
    \alt<2->{x : \m{T}{A}}{\phantom{x : \m{T}}\hspace{6.8pt}\G} \vdash M : B
    \\\pause\pause
    \begin{array}{rrcl}
      \text{modes} & T &\bnfeq& \id ~~|~~ \op ~~|~~ \iso ~~|~~ \path
    \end{array}
  \end{mathpar}
\end{frame}

\begin{frame}{1. Bottom-up mode inference: duplication}
  %% Talk here about always thinking of tones as an *output* of our
  %% type-checking judgment.
  \vspace{-1em}
  \begin{mathpar}
    \infer{x : \m{\hil T} A \vdash M : A \\ x : \m{\hil U} A \vdash N : B}{
      x : \m{\hil{\alt<2>{T \wedge U}{??????}}} A \vdash (M,N) : A \x B
    }\vspace{.5em}\pause

    \begin{tikzpicture}[scale=1.25,baseline=(current bounding box.center)]
      \node (top)  at ( 0, 1) {$\path$};
      \node (bot)  at ( 0,-1) {$\iso$};
      \node (-1)   at (-1, 0) {$\id$};
      \node (1)    at ( 1, 0) {$\op$};
      \draw (top) -- (-1) -- (bot) -- (1) -- (top);
    \end{tikzpicture}
  \end{mathpar}
\end{frame}

%% %% TODO: figure out how to explain \wedge
%% \begin{frame}{1. Bottom-up mode inference: $\wedge$}
%%   \LARGE
%%   \begin{mathpar}
%%     \begin{array}{rcl}
%%       T \wedge T &=& T\\
%%       T \wedge U &=& U \wedge T\\
%%       \iso \wedge T &=& \iso\\
%%       \id \wedge \op &=& \iso\\
%%       \path \wedge T &=& T
%%     \end{array}

%%     \begin{tikzpicture}[scale=1.4,baseline=(current bounding box.center)]
%%       \node (top)  at ( 0, 1) {$\path$};
%%       \node (bot)  at ( 0,-1) {$\iso$};
%%       \node (-1)   at (-1, 0) {$\id$};
%%       \node (1)    at ( 1, 0) {$\op$};
%%       \draw (top) -- (-1) -- (bot) -- (1) -- (top);
%%     \end{tikzpicture}
%%   \end{mathpar}
%% \end{frame}

\newcommand{\compx}[1]{{\alt<1>{\color{Green}}{}#1}}
\newcommand{\compy}[1]{{\alt<1>{\color{Orange}}{}#1}}
\newcommand{\comptone}[1]{{\alt<2>{\hilit}{}#1}}

\begin{frame}{1. Bottom-up mode inference: composition}
  \vspace{-1em}
  \[
  \infer{\compx{x} : \m{\comptone{T}} A \vdash \compx{M} : B
         \\ \compy{y} : \m{\comptone{U}} B \vdash \compy{N} : C}
        {\compx{x} : \m{\comptone{UT}} A
          \vdash \kw{let}~ \compy{y} = \compx{M} ~\kw{in}~ \compy{N}}
  \]\vspace{0em}

  \pause\pause\large
  {Composition $UT$ is a monoid, with $\id$ neutral:}

  \[\begin{array}{cr|cccc}
  \multicolumn{2}{c|}{\multirow{2}{*}{$UT$}}
  & \multicolumn{4}{c}{T}\\
  && \id & \op & \iso & \path\\\hline
  \multirow{4}{*}{$U$}
  & \id & \id & \op & \iso & \path\\
  & \op & \op & \id & \iso & \path\\
  & \iso & \iso & \iso & \iso & \path\\
  & \path & \path & \path & \iso & \path
  \end{array}\]
\end{frame}

\begin{frame}
  \normalsize
  Composition $UT$ and meet $T \wedge U$ form a semiring!\\
  Other systems with semiring-valued annotations:
  \begin{enumerate}
  \item \emph{Linear Haskell} (POPL 2018)
    %% TODO FIXME ARGH
  \item \emph{I Got Plenty o' Nuttin} (McBride)
  \item \emph{Monotonicity Types} (PMLDC 2017)
  \item \emph{Bounded Linear Types in a Resource Semiring} (ESOP 2014)
  \item \emph{The Next 700 Modal Type Assignment Systems} (TYPES 2015)
  \item \emph{A Fibrational Framework for Substructural and Modal Logics}
    (FSCD 2017)
    \textbf{massively} generalizes this pattern.
  \end{enumerate}
  ... more?
\end{frame}


%% 2. Modal subtyping
\newcommand{\subtype}{\mathrel{<:}}

\begin{frame}{2. Modal subtyping}
  \LARGE\vspace{-1em}
  \[{(\fnof{x} \kw{let}~\kw{box}~u = x~\kw{in}~
    u - \kw{box}\,(\log u)) : \isof{\R} \to \R}\]

  %% All of the things that explicitly convert into and out of boxes can be
  %% subsumed by implicit subtyping.
\end{frame}

%% \begin{frame}{2. Modal subtyping: \iso-elimination}
%%   \huge
%%   \[ \fnof{x} x : \isof{\R} \to \R \pause\iff \isof{\R} \subtype \R \]
%% \end{frame}

\begin{frame}{2. Modal subtyping: \iso-introduction?}
  \LARGE
  \[
  \infer[\rulename subsumption]
        {\G \vdash M : A \\ A \subtype B}
        {\G \vdash M : B}
  \]\vspace{0em}

  \centering But \(A \not\subtype \isof{A}\)!
\end{frame}

\begin{frame}{2. Modal subtyping!}
  \huge\vspace{-1em}
  \[\m{\hil U} A \subtype B\]
  \vspace{0em}

  \Large Finds the \emph{most general} mode $U$ such that $UA \subtype B$.
\end{frame}

\begin{frame}{2. Modal subtyping: \iso-introduction!}
  \vspace{-1em}
  \begin{mathpar}
  \infer[\rulename subsumption]
        {x : \m{\hil T} A \vdash M : B
          \\ \m{\hil U} B \subtype C}
        {x : \m{\hil{UT}} A \vdash M : C}
  % TODO: cancel previous hilights after pause
  \pause

  \infer[\rulename subtyping into \iso]
        {\m{\hil T} A \subtype B}
        {\m{\hil{\iso T}} A \subtype \isof{B}}
  \vspace{1em}\\\pause
  %
  \infer[\rulename \iso{}-introduction via subsumption]
        {x : \m{\hil T} A \vdash M : B
          %% \\ \m{\hil \iso} B \subtype \isof{B}
        }
        {x : \m{\hil{\iso T}} A \vdash M : \isof{B}}
  \end{mathpar}
\end{frame}

\begin{frame}{2. Modal subtyping: \iso-elimination}
  \large\vspace{-1em}
  \[ f : \pathof{A} \to B \iff f : A \to \isof{B} \]
  \pause

  Which suggests this elimination rule for $\iso$:
  \[
    \infer{x : \m{T} A \vdash M : \isof B}
          {x : \m{\pathof T} A \vdash \mathop{\kw{unbox}} M : B}
  \]\vspace{0pt}

  \pause
  Happily, we can do this with subtyping!
  \begin{mathpar}
    \infer[\rulename subtyping out of $\iso$]
          {\m{T} A \subtype B}
          {\m{T \path} \isof A \subtype B}

    \infer[\rulename \iso-elimination via subsumption]
          {x : \m{T} A \vdash M : \isof B}
          {x : \m{\path T} A \vdash M : B}
  \end{mathpar}
\end{frame}

\begin{frame}
  \textbf{How to infer monotonicity:}

  \begin{enumerate}
  \item Infer variable usage modes bottom-up.
  \item Use modal subtyping for implicit coercion.
  \end{enumerate}
\end{frame}

\begin{frame}{State of work}\vspace{-1em}
  \begin{enumerate}
  \item \textbf{Solved:} Pattern matching!
  \item \textbf{Conjectured:} Substitution.
  \item \textbf{Unsolved:} Internalizing $\path$ w/o annotations.
  \item \textbf{Open:} HM-style type inference?
  \item Pfenning-Davies elim rule is derivable from substitution.
  \end{enumerate}
  \vspace{1em}

  \centering
  \url{http://www.rntz.net/files/tones.pdf}\\
  \url{http://www.rntz.net/datafun}
\end{frame}


%% %% Pfenning-Davies vs tonal intro/elim rules.

%% \begin{frame}
%%   \Large
%%   \vspace{-1em}
%%   \begin{mathpar}
%%     \begin{array}{rcl}
%%       T &\bnfeq& 1 ~~|~~ \iso\\
%%       \G &\bnfeq& \varepsilon ~~|~~ \G, x : \m{T}{A}
%%     \end{array}
%%     \\
%%     \infer[\rulename Pfenning-Davies]
%%           {\{x : \m{\iso}{A} \in \G\} \vdash M : A}
%%           {\G \vdash \kw{box}~M : \iso{A}}

%%     \pause
%%     \infer[\rulename Ours]{\G \vdash M : A}{\disc{\G} \vdash M : \isof{A}}
%%   \end{mathpar}
%% \end{frame}

%% \begin{frame}\large
%%   \begin{mathpar}
%%     \infer[PfD \iso-intro]
%%           {\{\disc{A} \in \G\} \vdash A}
%%           {\G \vdash \isof{A}}

%%     \infer[PfD \iso-elim]
%%           {\G \vdash \isof{A} \\ \G, \disc{A} \vdash B}
%%           {\G \vdash B}
%%   \end{mathpar}

%%   \begin{mathpar}
%%     \infer[\iso-intro]{\G \vdash A}{\disc{\G} \vdash \isof{A}}

%%     \infer[\iso-elim]{\G \vdash \isof{A}}{\m{\path}{\G} \vdash A}
%%   \end{mathpar}
%% \end{frame}

%% %% The full Pfenning-Davies rules
%% \begin{frame}{Propagate variable usage bottom-up}
%%   \vspace{-1em}
%%   \begin{mathpar}
%%     \G \bnfeq \varepsilon ~~|~~ \G, x : A ~~|~~ \G, x : \disc{A}
%%     \\
%%     \infer{\{x : \disc{A} ~|~ x : \disc{A} \in \G\} \vdash M : A}
%%           {\G \vdash \kw{box}~M : \iso{A}}
       
%%     \infer
%%         {\G \vdash M : \isof{A} \\
%%           \G, x : \disc{A} \vdash N : B}
%%         {\G \vdash \kw{let}~\kw{box}~x = M ~\kw{in}~ N : B}
%%   \end{mathpar}
%% \end{frame}


%% %% ---- BIDIRECTIONAL INFERENCE ----
%% \newcommand{\inp}{\color{ACMPurple}}
%% \newcommand{\outp}{\color{Cyan}}
%% \newcommand{\inpu}[1]{{\inp{#1}^-}}
%% \newcommand{\outpu}[1]{{\outp{#1}^+}}
%% \newcommand{\checks}[3]{{\inp#1} \vdash {\inp#2} ~\mathbf{checks}~ {\inp#3}}
%% \newcommand{\infers}[3]{{\inp#1} \vdash {\inp#2} ~\mathbf{infers}~ {\outp#3}}

%% \begin{frame}{Input/output modes}
%%   %% Talk about algorithmic interpretation.
%%   \[
%%   \begin{array}{l@{\hskip 2em}l}
%%     \text{Na\"ive} & \inpu{\G} \vdash \inpu{M} : \inpu{A}
%%     \vspace{1em}\\\pause
%%     \text{Hindley-Milner} & \inpu{\G} \vdash \inpu{M} : \outpu{A}
%%     \vspace{1em}\\\pause
%%     \multirow{2}{*}{\text{Bidirectional}}
%%     &
%%     \checks{\inpu{\G}}{\inpu{m}}{\inpu{A}}\\
%%     & \infers{\inpu{\G}}{\inpu{e}}{\outpu{A}}
%%   \end{array}
%%   \]
%% \end{frame}

%% \begin{frame}{Bidirectional type inference}
%%   %% TODO: I should talk about input/output moding here!
%%   \[
%%   \begin{array}{llcl}
%%     \text{checking terms} & m &\bnfeq& \fnof{x} m ~~|~~ e\\
%%     \text{inferring terms} & e &\bnfeq& x ~~|~~ e\;m ~~|~~ (m : A)
%%   \end{array}
%%   \]

%%   %% I should not put all of these on one slide. I should probably ignore some.
%%   %% I only have ~25 mins.
%%   \begin{mathpar}
%%     \infer{\infers{\G}{e}{A \to B} \\ \checks{\G}{m}{A}}
%%           {\infers{\G}{e\;m}{B}}
%%     %% \infer{\checks{\G, x : A}{m}{B}}{\checks{\G}{\fnof{x} m}{A \to B}}
%%     %% \infer{\infers{\G}{e}{B}}{\checks{\G}{e}{B}}
%%     %% \infer{ }{\infers{\G,x : A}{x}{A}}
%%     %% \infer{\checks{\G}{m}{A}}{\infers{\G}{m : A}{A}}
%%   \end{mathpar}
%% \end{frame}


\end{document}
