\section{Introduction}
\label{sec:intro}


\paragraph{Contributions} The main contribution of this paper is
extending Datalog's semi-naive evaluation into a program transformation
which works on higher-order programs.
\begin{itemize}
\item We define a kernel language for the higher-order query language Datafun, and
  present its semantics in a categorical style, showing that its old semantics
  in terms of monotone function on partially ordered sets satisfies this presentation.
\item Next, we define a category of ``changeable posets'', which
  integrates the partially-ordered sets used to interpret Datafun with
  the \emph{change structures} of \citet{cai-et-al}. This gives us a semantic
  criterion for identifying programs which have an incrementalization. We then
  interpret the semantics of Datafun in changeable posets,
  and show that this new semantics is equivalent to the old
  one. This establishes that every well-typed Datafun program is
  incrementalizable.
\item Guided by this incremental semantics, we read off a program
  transformation to statically converts all well-typed Datafun
  programs (including ones that use higher-order functions) into an
  incrementalized version. We then use a logical relation to show that
  our incrementalization is correct, and so can be used to optimize
  Datafun programs.
\item We also implement our program transformation in a small compiler,
  and use this to show that we can automatically compile Datafun terms
  into optimized Haskell programs with the expected operational behaviour.
\end{itemize}



%%% Local Variables:
%%% TeX-master: "seminaive-datafun"
%%% End:
