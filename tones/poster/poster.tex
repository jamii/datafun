%% TODO:
%% - charter or palatino for body font
%% - understand layout commands
%% - figure out what goes on poster & where

% Based on the Gemini theme, https://github.com/anishathalye/gemini

\documentclass[final,dvipsnames]{beamer}

% ====================
% Packages
% ====================

\usepackage[T1]{fontenc}
%\usepackage{lmodern}
\usepackage{nccmath} %fix spacing issues. needs to load before beamerposter.

%% Paper sizes:
%\usepackage[size=a0, scale=1.25]{beamerposter}
%% 24 x 36" = 60.96 x 91.44cm
%\usepackage[size=custom, width=60.96, height=91.44, scale=1]{beamerposter}
%\usepackage[size=custom, height=60.96, width=91.44, scale=1]{beamerposter}
%% 36 x 48" = 91.44 x 121.92cm
\usepackage[size=custom, width=91.44, height=121.92, scale=1.3]{beamerposter}
%\usepackage[size=custom, height=91.44, width=121.92, scale=1.3]{beamerposter}

\usepackage{graphicx}
\usepackage{booktabs}
\usepackage{tikz}
%\usepackage{pgfplots}
\usepackage{mathpartir}
\usepackage{multirow}

%% Argh, these define commands so I have to include them.
\usetheme{gemini}
\usecolortheme{gemini}          %gemini, mit, labsix

%% \usepackage{PTSans,PTSerif}
%% \usepackage{fbb}
%% \usepackage{cochineal}
%% \usepackage{newpxtext}

%% Load newpxtext last so it gets used by eulervm.
%\usepackage{librebaskerville}
\usepackage[scaled=0.95]{sourcesanspro}
\usepackage{eulervm}

\newfontfamily\SSP{SourceSansPro}
\newfontfamily\Baskerville[Scale=0.855]{LibreBaskerville}
\newfontfamily\Charter[Scale=0.98]{XCharter}
\newfontfamily\Palatino[Scale=0.96]{TeXGyrePagellaX}
%% % consider bumping scale if I only use this for the title.
%% \newfontfamily{\PTSans}{PTSans}
%% \newfontfamily{\PTSerif}{PTSerif}
%% \newfontfamily{\FBB}{fbb}
%% \newfontfamily{\Cochineal}{Cochineal}
%% \newfontfamily{\Noto}[Scale=0.9]{Noto Sans}

\setbeamerfont{headline}{family=\Charter}
\setbeamerfont{block title}{family=\SSP}
\setbeamerfont{heading}{family=\SSP}
\setbeamerfont{block body}{family=\SSP}

%% \setbeamerfont{block title}{family=\Charter}
%% \setbeamerfont{block body}{family=\Charter}

\definecolor{labsixorange}{RGB}{243, 111, 33}
\definecolor{mitred}{cmyk}{0.24, 1.0, 0.78, 0.17}

\setbeamercolor{headline}{fg=lightgray,bg=darkblue}
\setbeamercolor{headline rule}{bg=gray}
\setbeamercolor{block title}{fg=darkblue,bg=white}
\setbeamercolor{block alerted title}{fg=mitred,bg=lightgray}


% ====================
% Lengths
% ====================

% If you have N columns, choose \sepwidth and \colwidth such that
% (N+1)*\sepwidth + N*\colwidth = \paperwidth
\newlength{\sepwidth}
\newlength{\colwidth}
% 2 columns
\setlength{\sepwidth}{0.03\paperwidth}
\setlength{\colwidth}{0.455\paperwidth}
%% % 3 columns
%% \setlength{\sepwidth}{0.025\paperwidth}
%% \setlength{\colwidth}{0.3\paperwidth}

\newcommand{\separatorcolumn}{\begin{column}{\sepwidth}\end{column}}

% ====================
% Title
% ====================

\title{Type inference for monotonicity}
\author{Michael Arntzenius}
\institute[shortinst]{University of Birmingham}


%% ========== Commands ==========

\newcommand\rulestyle{\sffamily\scshape}
\newcommand\rulename[1]{{\rulestyle#1}}
\newcommand{\catname}[1]{\textbf{#1}}
\newcommand{\Preorder}{\catname{Preorder}}

\newcommand\N{\mathbb{N}}
\newcommand\R{\mathbb{R}}
\newcommand\x\times
\newcommand\todo[1]{{\color{Red}#1}}
\newcommand\G\Gamma
\newcommand\D\Delta
\newcommand\mto{\overset{+}{\to}}

\newcommand{\opcolor}{\color{ForestGreen}}
\newcommand{\isocolor}{\color{NavyBlue}}
\newcommand{\pathcolor}{\color{Bittersweet}}

\newcommand{\id}{\mathrm{id}}
\newcommand{\op}{\mathrm{\opcolor op}}
\newcommand{\iso}{{\texorpdfstring{\ensuremath{\isocolor\Box}}{iso}}}
\renewcommand{\path}{{\texorpdfstring{\ensuremath{\pathcolor\lozenge}}{path}}}

\newcommand{\idof}{\id\,}
\newcommand{\opof}{\op\,}
\newcommand{\isof}{\iso}
\newcommand{\pathof}{\path}

%% TODO: remove these.
\newcommand{\cid}{\id}
\newcommand{\cop}{{\opcolor\op}}
\newcommand{\ciso}{{\isocolor\iso}}
\newcommand{\cpath}{{\pathcolor\path}}

\newcommand\subtype{\mathrel{<:}}
\newcommand\fname[1]{\ensuremath{\mathrm{#1}}}
\newcommand\fn\lambda
\newcommand\fnof[1]{\fn{#1}.~}
\newcommand\kw[1]{\fname{#1}}

\newcommand\toiso{\overset\ciso\to}
\newcommand\toid{\overset\cid\to}
\newcommand\toop{\overset\cop\to}

\newcommand\isovar[1]{\mathbold{#1}}


% ====================
% Body
% ====================

\begin{document}

\begin{frame}[t]
\begin{columns}[t]
\separatorcolumn



\begin{column}{\colwidth}

  \begin{block}{Problem: Ensuring functions are monotone}
    \strong{Monotonicity} crops up in a variety of interesting places:
    \begin{enumerate}
    \item In Datafun~\cite{datafun}, a functional query language generalising
      Datalog, monotonicity is needed to ensure fixed-point expressions
      terminate.

    \item In abstract interpretation, static analyses are phrased as monotone
      maps on lattices.

    \item For ensuring eventual consistency in distributed systems \cite{bloom}
      and determinism in concurrent systems \cite{lvars}.
    \end{enumerate}

    In all these contexts, it's useful to be able to guarantee a function is monotone. So: \strong{how can we ensure monotonicity using types?}

  \end{block}

  \begin{block}{Modes of (non-)monotonicity}
    I consider four \strong{modes}, ways a function may respect the order on its domain:
    \begin{itemize}
    \item $\cid$ is monotone, or order-preserving. For example, $\fnof{x} x$.
    \item $\cop$ is {\opcolor antitone}, or order-inverting. For example,
      $\fname{not} : \fname{bool} \to \fname{bool}$.
    \item $\ciso$ is {\isocolor equivariant}, preserving only equivalence.
      Usually, \emph{all} functions are equivariant.
    \item $\cpath$ is {\pathcolor bivariant}, or both mono- and antitone. For example, $\fnof{x} 42$.
    \end{itemize}

    Formally, modes alter \strong{preorderings} (reflexive, transitive
    relations), as shown in Figure~\ref{fig:mode-ops}. We say $f : A \to B$ has
    mode $T$ iff $f$ is monotone from $TA \to B$.

    %% TODO: diagrams for what each mode does to a specific ordering, like Kevin
    %% Clancy's. Maybe make sure that we have a fencepost-like example, for path?
  \end{block}
  
  \begin{block}{Approach 1: Annotate the arrows}
    Annotating function types with their mode is the obvious approach, used in Datafun~\cite{datafun} and variance typing \cite{DBLP:conf/csr/Abel06}:

    \[
      \begin{array}{l}
        \fname{setMap} ~:~
        (A \toiso B) \toiso \fname{Set}~A \toid \fname{Set}~B\\
        \fname{setMap}~f~xs ~=~ \kw{do}~x \leftarrow xs\\
        \phantom{\fname{setMap}~f~xs ~=~ \kw{do}~}\fname{return}~(f\;x)
      \end{array}
    \]

    But it cannot capture more complex input-output ordering relationships:

    \[
      \begin{array}{l}
        \fname{subtractEach} ~:~
        \fname{List}~(\N \x {\opcolor \N}) \overset{???}{\to} \fname{List}~\N
        \\
        \fname{subtractEach}~ xs ~=~ \fname{map}~(\fn (x,y).~ x-y) ~xs
      \end{array}
    \]

    A mode on the arrow cannot indicate this function is \strong{monotone} in
    the first half of each $(\N \x \N)$ pair, but \strong{\opcolor antitone} in
    the second.

    %% TODO: give subtractEach a red/yellow background \& setMap a green one.
    %% or, use green/red marker to put a check mark next to setMap & an X next
    %% to subtractEach.
    %%
    %% TODO: cite Datafun and variance typing as examples of this approach.
  \end{block}

  \begin{block}{Approach 2: Modal types}
    Let $(\opof A)$ be $A$ with its ordering inverted. The \strong{\opcolor
      antitone} maps $(A \toop B)$ are exactly the \strong{monotone} maps
    $(\opof A \to B)$. Instead of annotating arrows, we can make \strong{all}
    functions monotone, and apply modes directly to types! Now
    \fname{subtractEach} has a precise type:

    \[ \fname{subtractEach} ~:~ \fname{List}~(\N \x {\opof \N}) \to \fname{List}~\N \]

    A key feature of modal type systems \cite{jrml} is \strong{multiple typing
      contexts} — $\Delta$ for equivariant variables, $\Gamma$ for monotone —
    which are manipulated by the intro and elim rules for $\isof A$:
    \vspace{.5em}
    \begin{mathpar}
      \infer{\Delta;\emptyset \vdash M : A}
            {\Delta;\Gamma \vdash \kw{box}~M : \isof A}

      \infer{\Delta;\Gamma \vdash M : \isof A \\
      \Delta, \isovar{x} : A;\Gamma \vdash N : C}
      {\Delta;\Gamma \vdash \kw{let}~\kw{box}~\isovar{x} = M ~\kw{in}~N : C}
    \end{mathpar}

    However, needing to \strong{explicitly introduce and eliminate} modal types
    clutters up functions like \fname{setMap}:

    \[
      \begin{array}{l}
        \fname{setMap} ~:~
        \iso(\iso A \to B) \to \fname{Set}~A \to \fname{Set}~B\\
        \fname{setMap}~f~xs =\\
        \quad\kw{let}~\kw{box}~\isovar{g} = f ~\kw{in}\\
        \quad
        \kw{do}~x \leftarrow xs\\
        \quad\phantom{\kw{do}~}\kw{let}~\kw{box}~\isovar{y} = x\\
        \quad\phantom{\kw{do}~}
        \fname{return}~(\kw{box}~(\isovar{g}~(\kw{box}~\isovar{y})))
      \end{array}
    \]

    %% TODO: cite Pfenning-Davies in bibliography.

    %% TODO: show the Pf-D typing rules? explain Set A = Downset(\iso A)?
  \end{block}

  \begin{block}{Our approach: Modal subtyping!}
    \strong{Goal:} Handle functions which are monotone in only part of their
    input \emph{without} clunky term annotations.

    \strong{Method:} Construct and eliminate modal types \emph{implicitly} via \strong{subtyping}.
    %
    Since types are preorders, subtyping means \emph{subpreordering}:
    \begin{center}
      $A \subtype B$ iff $A \subseteq B$ and $x \le y : A \implies x \le y : B$
    \end{center}

    For example, $\isof A <: A$, because $x \le y \wedge y \le x : A \implies x
    \le y : A$. This lets us \emph{eliminate} $\isof A$ via subtyping.
  \end{block}

\end{column}

\separatorcolumn


\begin{column}{\colwidth}
  \begin{figure}
    \vspace{-1em}
    \begin{mathpar}
      \begin{array}{llc}
        a \le b : \idof A &\iff& a \le b : A\\
        a \le b : \opof A &\iff& a \ge b : A\\
        a \le b : \isof A &\iff& a \le b \wedge a \ge b : A\\
        a \le b : \pathof A &\impliedby& a \le b \vee b \le a : A
      \end{array}

      \begin{tikzpicture}[scale=2.5,baseline=(current bounding box.center)]
        \node (top)  at ( 0, 1) {$\cpath$};
        \node (bot)  at ( 0,-1) {$\ciso$};
        \node (-1)   at (-1, 0) {$\cid$};
        \node (1)    at ( 1, 0) {$\cop$};
        \draw (top) -- (-1) -- (bot) -- (1) -- (top);
      \end{tikzpicture}

      \begin{array}{cr|cccc}
        \multicolumn{2}{c|}{\multirow{2}{*}{$UT$}}
        & \multicolumn{4}{c}{T}\\
        && \cid & \cop & \ciso & \cpath\\\hline
        \multirow{4}{*}{$U$}
        & \cid & \cid & \cop & \ciso & \cpath\\
        & \cop & \cop & \cid & \ciso & \cpath\\
        & \ciso & \ciso & \ciso & \ciso & \cpath\\
        & \cpath & \cpath & \cpath & \ciso & \cpath
      \end{array}
    \end{mathpar}

    \caption{\ Modes, the mode lattice, and mode composition}
    \label{fig:mode-ops}
  \end{figure}

  \begin{block}{Problem: Introducing $\iso$ via subtyping}
    \todo{TODO: challenges? 1. how do you \emph{introduce} a box by subtyping?\\
    2. function application.\\
    3. pattern-matching (probably leave out).}
  \end{block}

  \begin{block}{Problem: Function application is ambiguous!}
    Consider applying a function $f : \iso(A \to B)$. Which of these two valid
    subtypings should we apply to determine the expected argument and return
    types?

    \begin{align*}
      \iso(A \to B) &\subtype A \to B & \text{because $\iso C \subtype C$ for any $C$}\\
      \iso(A \to B) &\subtype \iso A \to \iso B
      & \text{by functoriality of $\iso$}
    \end{align*}

    Neither type is more specific, and both usages occur in practice.
  \end{block}

  \begin{block}{Future work}
    \heading{Is modal subtyping useful elsewhere?}

    \heading{Hybridize with SMT-based approaches}
    \todo{LiquidHaskell and other lightweight verification techniques. Probably
      necessary for complex code, but can we verify simple code without
      appealing to an SMT solver? Future work: a hybrid approach.}
  \end{block}

  \begin{block}{References}
    \footnotesize{\bibliographystyle{plain}\bibliography{poster}}
  \end{block}

\end{column}


%% \separatorcolumn

%% \begin{column}{\colwidth}

%% \end{column}


\separatorcolumn
\end{columns}
\end{frame}

\end{document}
