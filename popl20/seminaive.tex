%% For double-blind review submission, w/o CCS and ACM Reference (max submission space)
\documentclass[acmsmall,review,anonymous]{acmart}\settopmatter{printfolios=true,printccs=false,printacmref=false}
%% For double-blind review submission, w/ CCS and ACM Reference
%\documentclass[acmsmall,review,anonymous]{acmart}\settopmatter{printfolios=true}
%% For single-blind review submission, w/o CCS and ACM Reference (max submission space)
%\documentclass[acmsmall,review]{acmart}\settopmatter{printfolios=true,printccs=false,printacmref=false}
%% For single-blind review submission, w/ CCS and ACM Reference
%\documentclass[acmsmall,review]{acmart}\settopmatter{printfolios=true}
%% For final camera-ready submission, w/ required CCS and ACM Reference
%\documentclass[acmsmall]{acmart}\settopmatter{}


%% Journal information
%% Supplied to authors by publisher for camera-ready submission;
%% use defaults for review submission.
\acmJournal{PACMPL}
\acmVolume{1}
\acmNumber{CONF} % CONF = POPL or ICFP or OOPSLA
\acmArticle{1}
\acmYear{2018}
\acmMonth{1}
\acmDOI{} % \acmDOI{10.1145/nnnnnnn.nnnnnnn}
\startPage{1}

%% Copyright information
%% Supplied to authors (based on authors' rights management selection;
%% see authors.acm.org) by publisher for camera-ready submission;
%% use 'none' for review submission.
\setcopyright{none}
%\setcopyright{acmcopyright}
%\setcopyright{acmlicensed}
%\setcopyright{rightsretained}
%\copyrightyear{2018}           %% If different from \acmYear

%% Bibliography style
\bibliographystyle{ACM-Reference-Format}
%% Citation style
%% Note: author/year citations are required for papers published as an
%% issue of PACMPL.
\citestyle{acmauthoryear}   %% For author/year citations


%%%%%%%%%%%%%%%%%%%%%%%%%%%%%%%%%%%%%%%%%%%%%%%%%%%%%%%%%%%%%%%%%%%%%%
%% Note: Authors migrating a paper from PACMPL format to traditional
%% SIGPLAN proceedings format must update the '\documentclass' and
%% topmatter commands above; see 'acmart-sigplanproc-template.tex'.
%%%%%%%%%%%%%%%%%%%%%%%%%%%%%%%%%%%%%%%%%%%%%%%%%%%%%%%%%%%%%%%%%%%%%%


%% Some recommended packages.
\usepackage{booktabs}   %% For formal tables:
                        %% http://ctan.org/pkg/booktabs
\usepackage{subcaption} %% For complex figures with subfigures/subcaptions
                        %% http://ctan.org/pkg/subcaption

\newcommand\naive{na\"ive}
\newcommand\Naive{Na\"ive}


\begin{document}

%% Title information: \title[Short Title]{Full Title}
%% Short Title is optional; when present, will be used in header instead of Full Title.
\title{Semi\naive{} Evaluation for a Higher-Order Functional Language}
% Optional: \subtitle{}, \titlenote{}, \subtitlenote{}

%% Author information
%% Contents and number of authors suppressed with 'anonymous'.
%% Each author should be introduced by \author, followed by
%% \authornote (optional), \orcid (optional), \affiliation, and
%% \email.
%% An author may have multiple affiliations and/or emails; repeat the
%% appropriate command.
%% Many elements are not rendered, but should be provided for metadata
%% extraction tools.

%% Author with single affiliation.
\author{Michael Arntzenius}
%\authornote{with author1 note}
%\orcid{nnnn-nnnn-nnnn-nnnn}             %% \orcid is optional
\affiliation{
  %\position{Position1}
  \department{School of Computer Science}              %% \department is recommended
  \institution{University of Birmingham}            %% \institution is required
  %\streetaddress{Street1 Address1}
  \city{Birmingham}
  %\state{State1}
  \postcode{B15 2TT}
  \country{United Kingdom}                    %% \country is recommended
}
\email{daekharel@gmail.com}          %% \email is recommended

%% Author with two affiliations and emails.
\author{Neelakantan R. Krishnaswami}
%\authornote{with author2 note}          %% \authornote is optional;
%\orcid{nnnn-nnnn-nnnn-nnnn}             %% \orcid is optional
\affiliation{
  %\position{Position2a}
  \department{Department of Computer Science and Technology} %% \department is recommended
  \institution{University of Cambridge}           %% \institution is required
  %\streetaddress{Street2a Address2a}
  \city{Cambridge}
  %\state{State2a}
  \postcode{CB2 1TN}
  \country{United Kingdom}                   %% \country is recommended
}
\email{first2.last2@inst2a.com FIXME FIXME}         %% \email is recommended

%% Abstract
%% Note: \begin{abstract}...\end{abstract} environment must come
%% before \maketitle command
\begin{abstract}
Text of abstract \ldots.
\end{abstract}

%% 2012 ACM Computing Classification System (CSS) concepts
%% Generate at 'http://dl.acm.org/ccs/ccs.cfm'.
\begin{CCSXML}
<ccs2012>
<concept>
<concept_id>10011007.10011006.10011008</concept_id>
<concept_desc>Software and its engineering~General programming languages</concept_desc>
<concept_significance>500</concept_significance>
</concept>
<concept>
<concept_id>10003456.10003457.10003521.10003525</concept_id>
<concept_desc>Social and professional topics~History of programming languages</concept_desc>
<concept_significance>300</concept_significance>
</concept>
</ccs2012>
\end{CCSXML}

\ccsdesc[500]{Software and its engineering~General programming languages}
\ccsdesc[300]{Social and professional topics~History of programming languages}
%% End of generated code

%% Keywords
%% comma separated list
\keywords{keyword1, keyword2, keyword3}  %% \keywords are mandatory in final camera-ready submission

%% \maketitle
%% Note: \maketitle command must come after title commands, author
%% commands, abstract environment, Computing Classification System
%% environment and commands, and keywords command.
\maketitle


\section{Introduction}

Text of paper \ldots


%% Acknowledgments
\begin{acks}                            %% acks environment is optional
% TODO: acknowledge
% - Paolo Giarrusso, for discussion on incremental lambda calc.

% - Michael Peyton-Jones and possibly other Semmle folk, for discussion of
%   seminaive evaluation.

% - Max S New for help with the question "how do I prove my semantics are
%   naturally isomorphic without inducting on typing derivations"?

% - Achim Jung, for pointing out that to prevent nesting fixed points we'd need
%   a modal type system, which eventually led to the realization about
%   reinterpreting \iso.

  %%                                       %% contents suppressed with 'anonymous'
  %% %% Commands \grantsponsor{<sponsorID>}{<name>}{<url>} and
  %% %% \grantnum[<url>]{<sponsorID>}{<number>} should be used to
  %% %% acknowledge financial support and will be used by metadata
  %% %% extraction tools.
  %% This material is based upon work supported by the
  %% \grantsponsor{GS100000001}{National Science
  %%   Foundation}{http://dx.doi.org/10.13039/100000001} under Grant
  %% No.~\grantnum{GS100000001}{nnnnnnn} and Grant
  %% No.~\grantnum{GS100000001}{mmmmmmm}.  Any opinions, findings, and
  %% conclusions or recommendations expressed in this material are those
  %% of the author and do not necessarily reflect the views of the
  %% National Science Foundation.
\end{acks}


%% Bibliography
%\bibliography{bibfile}

%% %% Appendix
%% \appendix
%% \section{Appendix}

%% Text of appendix \ldots

\end{document}
